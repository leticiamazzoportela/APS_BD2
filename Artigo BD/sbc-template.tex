\documentclass[12pt]{article}

\usepackage{sbc-template}
\usepackage[utf8]{inputenc}
\usepackage[brazil]{babel}
\usepackage{graphicx,url}
%\usepackage[brazil]{babel}   
\usepackage[latin1]{inputenc}
     
\sloppy

\title{Uma análise comparativa da técnica de \textit{Tuning} nos SGBDs MySQL e PostgreSQL}

\author{Cláudia L. Poiet Sampedro\inst{1}, Jonas Felipe Alves\inst{2}, Letícia Mazzo Portela\inst{3}}

\address{\inst{,}\inst{2}\inst{,}\inst{3}Universidade Tecnológica Federal do Paraná - UTFPR \\ \ \ Departamento Acadêmico de Computação - DACOM \\ Coordenação de Ciência da Computação - COCIC}


\begin{document} 

\maketitle

%\begin{abstract}
  
%\end{abstract}
     
%\begin{resumo} 
  
%\end{resumo}


\section{Apresentação das Tecnologias}
    \subsection{MySQL}
        MySQL é um Sistema Gerenciador de Banco de Dados (SGBD), de código aberto, baseado em SQL (structured query language). Criado em 1995 e desenvolvido nas linguagens C e C++, este está disponível em 2 versões: MySQL Community Server e Enterprise Server, cujas diferenças são o custo e os plugins disponíveis [Rouse 2013].
        
        Este SGBD roda virtualmente em todas as plataformas, incluindo Linux, Unix e Windows. Entretanto o MySQL é comumente associado a aplicações web, utilizando alguma plataforma de desenvolvimento web, integrada com certo servidor [MySQL 2017].
        
        Sua evolução é clara quando comparadas as versões mais antigas com a atual. Funções como \textit{rollback}, níveis de isolamento de transações e \textit{commit} foram implementadas com o tempo.
    \subsection{PostgreSQL}
        O PostgreSQL é um SGBD lançado em 1989 e gerenciado pela PostgreSQL Global Development Group. O mesmo é relacional, gratuito e de código aberto, além de possuir alta performance, recursos avançados, fácil administração e ser amplamente utilizado no ambiente web [Souza et.al].
        
    	A seguir, estão expostas algumas das principais características do referido SGBD, segundo Vale (2013):
            \begin{itemize}
                \item Objeto Relacional (cada tabela define uma classe);
                \item Operadores e funções polimórficas;
                \item Compilação padronizada;
                \item Diversidade de linguagens procedurais, bem como PLPG/SQL, PL/Java, PL/PHP, PL/Python, dentre outras;
                \item Tipos de dados únicos.
            \end{itemize} 
    \subsection{Características}
         O MySQL possui foco na administração, e no baixo consumo de recursos do hardware, se tornando relativamente rápido. O mesmo, embora seja software livre, não possui uma licença totalmente livre, fazendo com que certas funcionalidades estejam disponíveis apenas na versão paga.
         
         O PostgreSQL possui foco nas funcionalidades, não possuindo grande preocupação com velocidade, mas, em contrapartida, este é munido de recursos mais complexos. Como sua licença é menos rigorosa, usuários podem fazer modificações caso julguem necessárias.  

\nocite{msql:33}
\nocite{postgre:37}
\nocite{postgre2:40}
\nocite{lamp:49}

\bibliographystyle{sbc}
\bibliography{sbc-template}

\end{document}
